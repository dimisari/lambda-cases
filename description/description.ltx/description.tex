\documentclass{article}

\usepackage[utf8]{inputenc}
\usepackage{amsmath}
\usepackage{fancyvrb}
\usepackage{graphicx}
\usepackage{syntax}
\usepackage{array}
\usepackage[left=2cm,right=2cm,top=2cm]{geometry}

\date{}
\author{
  Dimitris Saridakis
}

\def\imgs{../Images}
\def\H{Haskell}
\def\ra{\texttt{->}\ }
\def\Ra{\texttt{=>}\ }
\def\pend{\mbox{} \\\\}
%\newcommand\_[_]{_}

\renewcommand{\arraystretch}{1.5}

\begin{document}

\title{
\textbf{Lambda Cases (lcases)}
}
\maketitle

%\setcounter{tocdepth}{4}
%\setcounter{secnumdepth}{4}
\tableofcontents

\section{Introduction}

\H\ is a delightful language. Yet, for some reason, it doesn't seem to have it's 
rightful place in terms of popularity in industry. Why is it so?
Is it inherently hard to learn
and therefore only the brave enough students and corporations dare to use it, or
could it be that the syntax is perplexing to the amateur eye? It is my belief that 
with some syntax changes that give a greater familiarity to the new user, there
would be no language more compelling than (the new) \H. In an attempt to achieve
that familiarity, I present some new syntax, of which
some is closer to the imperative/OOP style (to attract more already experienced
programmers from these languages), some is closer to mathematics (in which most 
programmers should be experienced) and some is closer to natural language
(in which we are all already experienced). 

\section{Language Description}

An lcases program consists of a set of value, type and predicate definitions along
with type theorems. The "main" value determines the program's behaviour.
Constants and functions are all considered values and they have no real distinction
other than the fact that functions have a function type and constants don't.
Functions (just like "values") can be passed to other functions as arguments or can
be returned as a result of other functions. 

\paragraph{Program example: extended euclidean alogirthm}
\begin{verbatim}
// type definitions

tuple_type PrevCoeffs
value (prev_prev, prev : Int, Int)

tuple_type GcdAndCoeffs
value (gcd, a, b : Int, Int, Int)

// algorithm

extended_euclidean: (Int, Int) -> GcdAndCoeffs
  = (init_a_coeffs, init_b_coeffs) ==> ee_recursion

init_a_coeffs, init_b_coeffs: all PrevCoeffs
  = (1, 0), (0, 1)

ee_recursion: (PrevCoeffs, PrevCoeffs, Int, Int) -> GcdAndCoeffs
  = (a_coeffs, b_coeffs, x, cases) ->
    0 -> (x, a_coeffs.prev_prev, b_coeffs.prev_prev)
    y ->
      ee_recursion(next <== a_coeffs, next <== b_coeffs, y, x ==> mod <== y)
      where
      next: PrevCoeffs -> PrevCoeffs
        = fields -> (prev, prev_prev - x ==> div <== y * prev)

// reading, printing and main

read_two_ints : (Int x Int)WithIO
  = print <== "Please give me 2 ints";
    get_line >>= split_words o> apply(from_string)to_all o> ints ->
    ints ==> length ==> cases ->
      2 -> ints ==> with_env
      ... -> io_error <== "You didn't give me 2 ints"
 
print_gcd_and_coeffs : GcdAndCoeffs -> (Empty)WithIO
  = fields -> print("Gcd: " + gcd + "\nCoefficients: a = " + a + ", b = " + b)

main : (Empty)WithIO
  = read_two_ints >>= ints ->
    extended_euclidean(ints.1st, ints.2nd) ==> print_gcd_and_coeffs
\end{verbatim}

\subsection{Values}

\subsubsection{Literals}

Literals are the same as haskell \\ \\
\begin{tabular}{ |c|c| } 
\hline
Examples & Type \\ 
\hline
\hline
1, 2, 17, 42, -100 & Int \\ 
\hline
1.61, 2.71, 3.14, -1234.567 & Real \\ 
\hline
'a', 'b', 'c', 'x', 'y', 'z', '.', ',', '\textbackslash n' & Char \\
\hline
[1, 2, 3], ['a', 'b', 'c'], [1.61, 2.71, 3.14 ] &
ListOf(Int)s, ListOf(Char)s, ListOf(Real)s \\
\hline
"Hello World!", "What's up, doc?", "Alrighty then!" & String \\
\hline
\end{tabular}

\subsubsection{Identifiers}

An identifier is a string of lower case letters or underscore. 
\paragraph{Grammar}
\begin{grammar}
<identifier> ::= ( [a-z_] )*
\end{grammar}

\subsubsection{Operators}

\begin{tabular}{ |c|c|c|c| } 
\hline
Operator & Type & Description & Associativity \\ 
\hline
\hline
\texttt{==>} & (A, A \ra B) \ra B & Right function application & Left \\
\hline
\texttt{<==} & (A \ra B, A) \ra B & Left function application & Left \\
\hline
\texttt{o>} & (A \ra B, B \ra C) \ra (A \ra C) & Right function composition &
Left \\
\hline
\texttt{<o} & (B \ra C, A \ra B) \ra (A \ra C) & Left function composition &
Right \\
\hline
\texttt{\^} & (A)ToThe(B)Gives(C) \Ra (A, B) \ra C & General exponentiation &
Right \\
\hline
\texttt{*} & (A)And(B)MultiplyTo(C) \Ra (A, B) \ra C & General multiplication &
Left \\
\hline
\texttt{/} & (A)Divides(B)To(C) \Ra (A, B) \ra C & General division & Left \\
\hline
+ & (A)And(B)AddTo(C) \Ra (A, B) \ra C & General addition & Left \\ 
\hline
- & (A)SubtractsFrom(B)To(C) \Ra (B, A) \ra C & General subtraction & Left \\
\hline
= /= & (A)HasEquality \Ra (A, A) \ra Bool & Equality operators & None \\
\hline
\texttt{> < >= <=} & (A)HasOrder \Ra (A, A) \ra Bool & Order operators & None \\
\hline
\texttt{\& |} & (Bool, Bool) \ra Bool & Boolean operators & Left \\
\hline
\texttt{>>=} & (E)IsAnEnvironment \Ra (E(A), A \ra E(B)) \ra E(B) &
Monad bind & Left \\
\hline
\texttt{;} & (E)IsAnEnvironment \Ra (E(A), E(B)) \ra E(B) &
Monad then & Left \\
\hline
\end{tabular}

\subsubsection{Expressions}

\paragraph{Examples}

\begin{verbatim}
42

x

funny_identifier 

[1, 2, 3]

"Hello world!"

1.61 * 2.71 + 3.14

a -> 17 * a + 42

(x, y, z) -> (x^2 + y^2 + z^2)^(1/2)

n==>(+ 1)==>(^2)==>(* 3)==>print

f(x, y, z) + g(1, 2, 3)
\end{verbatim}

\paragraph{Description}\pend
The base of expressions, are literals and identifiers, those can be combined either
with operators, or by normal function application with mathematical notation. 
Finally, on top of that there can be added one of more abstractions (parameters)
in the beginning of the expressions with an arrow.

\paragraph{Grammar}\pend

\subsubsection{Definitions}

\paragraph{Examples}

\begin{verbatim}
foo : Int
  = 42

val1, val2, val3 : Int, Bool, Char
  = 42, true, 'a'

int1, int2, int3 : all Int
  = 1, 2, 3

succ : Int -> Int
  = x -> x + 1

f : (Int, Int, Int) -> Int
  = (a, b, c) -> a + b * c
\end{verbatim}

\paragraph{Description}\pend
To define a new value you give it a name, a type and an expression. It is possible
to group value definitions by seperating the names, the types and the expressions
with commas. It is also possible to use the keyword "all" to give the same type
to all the values.

\paragraph{Grammar}
\begin{grammar}
<value-definitions> ::=
<identifiers> `\ :\ ' (<types> | `all' <type>) `\\n\ \ =' <value-expressions>

<identifiers> ::= <identifier> ( `,\ ' <identifier> )*  

<types> ::= <type> ( `,\ ' <type> )*  

<value-expressions> ::= <value-expression> ( `,\ ' <value-expression> )*  
\end{grammar}


\paragraph{Abstractions}

\begin{verbatim}
x -> body
(x, y, z) -> body
cases -> body
(x, cases, z) -> body
\end{verbatim}

\subsection{Types}

\subsubsection{Type expressions}

\paragraph{Examples}

\begin{verbatim}
Int

String -> String 

Int x Int 

Int x Int -> Real

A -> A

(A -> B, B -> C) -> (A -> C)

((A, A) -> A, A, ListOf(A)s) -> A

((B, A) -> B, B, ListOf(A)s) -> B

(T)HasStringRepresantion => T -> String
\end{verbatim}

\paragraph{Description}\pend
\begin{tabular}{ |c|c| } 
\hline
Examples & Description \\ 
\hline
\hline
Int & \\
Char & Base types \\
String & \\ 
\hline
A \ra A &
Polymorphic types. A, B, C ... are type variables
\\
(A \ra B, B \ra C) \ra (A \ra C) &
\\ 
\hline
\end{tabular}

\paragraph{Differences from Haskell}\pend
\begin{tabular}{ |c|c|c| } 
\hline
lcases & haskell & difference description \\ 
\hline
\hline
A \ra A & a \ra a & Type variables for polymorphic types are  \\ 
\hline
\end{tabular}

\paragraph{Grammar}
\begin{grammar}
<type> ::= <type-application> | <product-type> | <function-type> \\

<type-application> ::=
[ <types-in-paren> ] <type-identifier> (<types-in-paren> ( [A-Za-z] )*)* [ <types-in-paren> ] 

<types-in-paren> ::= `(' <type> (`, ' <type>)* `)'

<type-identifier> ::= [A-Z] ( [A-Za-z] )* \\ 
 
<product-type> ::= <product-subtype> ( `\ x\ ' <product-subtype> )+

<product-subtype> ::=
`(' ( <function-type> | <product-type> ) `)' | <type-application> \\

<function-type> ::= <input-types-expression> `\ ->\ ' <one-type>

<input-types-expression> ::= <one-type> | <two-or-more-types-in-paren>

<one-type> ::= <type-application> | <product-type> | `(' <function-type> `)'

<two-or-more-types-in-paren> ::=  `(' <type> (`, ' <type>)+ `)'
\end{grammar}

\subsubsection{Tuple Types}

\paragraph{Definition Examples}

\begin{verbatim}
tuple_type Name
value (first_name, last_name) : String x String

tuple_type ClientInfo
value (name, age, nationality) : Name x Int x String

tuple_type Date
value (day, month, year) : Int x Int x Int

tuple_type (A)And(B)
value (a_value, b_value) : A x B

tuple_type (ExprT)WithPosition
value (expr, line, column) : ExprT x Int x Int
\end{verbatim}

\paragraph{Usage Examples}

\begin{verbatim}
giorgos_info: ClientInfo
  = (("Giorgos", "Papadopoulos"), 42, "Greek")

john_info: ClientInfo
  = (("John", "Doe"), 42, "American")

name_to_string: Name -> String
  = fields -> "First Name: " + first_name + "\nLast Name: " + last_name

print_name_and_nationality : ClientInfo -> (Empty)WithIO
  = fields -> print(name ==> name_to_string + "\nNationality: " + nationality)

print_error_in_expr : (SomeDefinedExprT)WithPosition -> (Empty)WithIO
  = ewp ->
    print(
      "Error in the expression:" + es +
      "\nAt the position: (" + ls + ", " + cs + ")"
    )
    where
    es, ls, cs : all String
      = ewp.expr==>to_string, ewp.line==>to_string, ewp.column==>to_string
\end{verbatim}

\paragraph{Description}\pend
Tuple types group many values into a single value. They are specified by their name,
the names of their subvalues and the types of their subvalues. They generate 
projection functions for all of their subvalues by using a '.' before the name of 
the subvalue. For example the ClientInfo type above generates the following 
functions:
\begin{verbatim}
.name : ClientInfo -> String
.age : ClientInfo -> Int
.nationality : ClientInfo -> String
\end{verbatim}
These functions shall be named "postfix functions" as the can just be appended to
their argument.

\paragraph{Definition Grammar}
\begin{grammar}
<tuple-type-definition> ::= ""\\
`tuple_type\ ' <type-application>
`\\nvalue\ ' `(' <identifier> (`,\ ' <identifier>)* `)' `\ :\ ' <product-type>
\end{grammar}

\subsubsection{Or Types}

\paragraph{Examples}

\begin{verbatim}
or_type Bool
values true | false

or_type Possibly(A)
values the_value:A | no_value

or_type ListOf(A)s
values non_empty:HeadAndTailOf(A)s | empty

tuple_type HeadAndTailOf(A)s
value (head : A, tail : ListOf(A)s)

is_empty : ListOf(A)s -> Bool
  = cases -> 
    empty -> true
    non_empty:anything -> false

get_head : ListOf(A)s -> Possibly(A)
  = cases -> 
    empty -> no_value
    non_empty:list -> the_value:list.head
\end{verbatim}

\paragraph{Description}\pend
Values of an or\_type are one of many cases that possibly have other values inside.
The cases which have other values inside are followed by a semicolon and the 
type of the internal value. The same syntax can be used for matching that particular 
case in a fucntion using the "cases" syntax, with the difference that after the
colon, we write the name given to the value inside. 
Or\_types and basic types are the only types on which
the "cases" syntax can be used. The cases of an or\_type which have a value
inside create functions. For example, the case "non\_empty" of a list creates the
function "non\_empty:" for which we can say:
\begin{verbatim}
non_empty: : HeadAndTailOf(A)s -> ListOf(A)s
\end{verbatim}
Similarly:
\begin{verbatim}
the_value: : A -> Possibly(A)
\end{verbatim}
These functions shall be named "prefix functions" as they are prepended to their
argument.
For example:
\begin{verbatim}
head_and_tail : HeadAndTailOf(Int)s
  = (1, [2, 3, 4])

list : ListOf(Int)s
  = non_empty:head_and_tail
\end{verbatim}
These functions can be used like any other function as arguments to other functions.
For example:
\begin{verbatim}
heads_and_tails_to_lists : ListOf(HeadAndTailOf(A)s)s -> ListOf(ListOf(A)s)s
  = apply(non_empty:)to_each
\end{verbatim}

\paragraph{Definition Grammar}
\begin{grammar}
<or-type-definition> ::= ""\\
`or_type\ ' <type-application> 
`\\nvalues\ ' <identifier> [ `:' <type> ] ( `\ |\ ' <identifier> [ `:' <type> ])*
\end{grammar}

\subsection{Type Logic}

\subsubsection{Type Predicate}

\subsubsection{Type Theorem}

\subsection{Grammar}
\subsubsection{Tokens}

\paragraph{Keywords}

\begin{verbatim}
cases use_fields tuple_type or_type
\end{verbatim}

\paragraph{Value names}

\paragraph{Type names}

\subsubsection{Core Grammar}

\setlength{\grammarparsep}{20pt}
\setlength{\grammarindent}{12em}

\paragraph{Program}
\hspace{1cm}\\
\begin{grammar}

<program> ::= (<value-definitions> | <type-def>)+

<value-definitions> ::=
<identifiers> `\ :\ ' (<types> | `all' <type>) `\\n\ \ =' <value-expressions>

<identifiers> ::= <identifier> ( `,\ ' <identifier> )*  

<types> ::= <type> ( `,\ ' <type> )*  

<value-expressions> ::= <value-expression> ( `,\ ' <value-expression> )*  

\end{grammar}
\hspace{1cm}\\

\paragraph{Types}

\hspace{1cm}\\


\hspace{1cm}\\

\paragraph{Value Expressions}

\hspace{1cm}\\
\begin{grammar}

<value-expression> ::= [ <input-expr> ] <cases-or-where> | <op-expr>

<cases-or-where> ::= <cases-expr> | <where-expr>

<where-expr> ::=
`let' <spicy-nl> (<value-definitions> <spicy-nls>)+ `in' <value-expression> <spicy-nl>

<cases-expr> ::= `cases' ( <case> )+ [ <default-case> ]

\end{grammar}

\section{Parser implimentation}

The parser was implemented using the parsec library.

\subsection{AST Types}

\subsection{Parsers}

\section{Translation to \H}
\section{Running examples}
\section{Conclusion}

\section{To be removed or incorporated}

Addition/Subtraction:
\begin{verbatim}
+ : (A)HasAddition => (A, A) -> A
- : (A)HasSubtraction => (A, A) -> A
\end{verbatim}
Equality and ordering:
\begin{verbatim}
= : (A)HasEquality => (A, A) -> Bool
<= : (A)HasOrder => (A, A) -> Bool
>= : (A)HasOrder => (A, A) -> Bool
\end{verbatim}

(fmap)\texttt{<inside>} | (W)IsAWrapper \Ra (A \ra B, W(A)) \ra W(B) | Apply inside operator \\
\texttt{(<*>)<wrapped_inside>} | (W)IsAWrapper \Ra (W(A \ra B), W(A)) \ra W(B) | Order operators \\

better as postfix functions \\
\hspace{1cm}\\
\paragraph{Examples in \H}

\begin{verbatim}
data ClientInfo =
  ClientInfoC String Int String

data WithPosition a = 
  WithPositionC a Int Int

data Pair a b = 
  PairC a b
\end{verbatim}

\paragraph{Examples in \H}

\begin{verbatim}
{-# language LambdaCase #-}

data Bool =
  Ctrue | Cfalse

data Possibly a =
  Cwrapper a | Cnothing

data ListOf_s a =
  Cnon_empty (NonEmptyListOf_s a) | Cempty

data NonEmptyListOf_s a =
  CNonEmptyListOf_s a (ListOf_s a)

is_empty :: ListOf_s a -> Bool
is_empty = \case
  Cempty -> Ctrue
  Cnon_empty (CNonEmptyListOf_s head tail) -> Cfalse

get_head :: ListOf_s a -> Possibly a
get_head = \case
  Cempty -> Cnothing
  Cnon_empty (CNonEmptyListOf_s head tail) -> Cwrapper head
\end{verbatim}

\paragraph{Examples in \H}
\begin{verbatim}
foo :: Int
foo = 42

val1 :: Int
val1 = 42
val2 :: Bool
val2 = true
val3 :: Char
val3 = 'a'

int1 :: Int
int1 = 1
int2 :: Int
int2 = 2
int3 :: Int
int3 = 3

succ :: Int -> Int
succ = \x -> x + 1

f :: Int -> Int -> Int -> Int
f = \a b c -> a + b * c
\end{verbatim}

Or Types the following have automatically generated functions:

\begin{verbatim}
is_case:
\end{verbatim}
%\newpage

%\paragraph{Examples}
%
%\begin{verbatim}
%\end{verbatim}

%\paragraph{Description}\pend
%desc
%
%\paragraph{Examples in \H}
%
%\begin{verbatim}
%\end{verbatim}

% \begin{tabular}{ |c|c| } 
% \hline
% 1 & 2 \\ 
% \hline
% \hline
% 1.1 & 1.2 \\ 
% \hline
% \end{tabular}

%  \includegraphics[width=10cm, height=8cm]{../Images/image.png}
\begin{syntdiag}
<ident> ‘(’
\begin{rep} \begin{stack} \\
<type> \begin{stack} \\ <ident> \end{stack}
\end{stack} \\ ‘,’ \end{rep}
\begin{stack} \\ ‘...’ \end{stack} ‘)’
\end{syntdiag}
\end{document}
